\section{Lightweight Directory Access Protocol (LDAP)}

The IRIX UNS includes an LDAP client protocol library that
enables clients to use this protocol to resolve name service requests.
LDAP is a directory service similar to X.500, but it requires less resources
for its implementation. Additionally, LDAP runs over the well-known 
TCP/IP and UDP protocols.
Under this scheme, the user can store information, which is generally 
kept in configuration files, in an LDAP database and access it
through the UNS LDAP client library. In order to do this, the user
needs to define a database schema which specifies how information will
be organized in, and retrieved from the LDAP directory. An LDAP client
library must know this schema in order to retrieve any information
from the LDAP directory database. Since no standard
exist regarding the schema that must be used to store regular Unix
configuration files in an LDAP directory, the IRIX LDAP client library
has been made configurable so that it can adapt to any schema the user
may be interested in.
A piece of the LDAP client configuration file must be dedicated to
each domain in which the LDAP protocol library will be used (i.e. as
dictated in the nsswitch.conf files).
The LDAP configuration file specifies which are the LDAP servers to be
used, parameters to conduct the search on each server, how to map each
name service request to an LDAP search filter\footnote{A search filter
is a string defining the value of one or more attributes that must
match the values of those same attributes in entries returned as a
result of the search operation.}, and how to transform an LDAP
response into its file format representation. 
With this type of flexibility the user can employ virtually any type of
schema he wish for storing the configuration information, including
the one we use as default and currently being discussed by the IETF as
a possible standard for this application.

\subsection{LDAP Client Configuration File}

The LDAP client configuration is stored in the following file:

\begin{center}
/etc/ldap-ns.conf
\end{center}

The information in the file /etc/ldap-ns.conf represents the
database schema used to store naming information for each domain. This
configuration file is read when the library is initialized or every
time the name server daemon receives the SIGHUP signal.
The configuration file has a part dedicated to each domain defined
by the user. Each domain section is composed of two parts: the server
specification part and the schema definition part also known as the name
service mapping part. The server specification part tells the LDAP client
library which servers to contact and what parameters to use while
resolving a name service query. The name service mapping (or database
schema) part of each domain section specifies how to map a name service
query into a LDAP query and it also specifies how to present the
retrieved information to the user.
Parsing of keywords in the configuration file is case-insensitive and
the maximum size of a line is of 1024 octets.
All keywords and options supported are specified in the next subsections.

\subsubsection{Comments}

Comments may appear anywhere in the configuration file and they must
be preceded by a {\bf ;} or {\bf \#} characters. Comments may
constitute a whole line or they may appear at the end of any keyword line
described in this section. The scope of the comment character, {\bf ;}
and {\bf \#}, is limited to one line.

\subsubsection{Debugging Level Specification}

The format of a domain specification line is as follows:

\begin{center}
{\bf 'trace'} $[$ {\bf 'f'} $]$ $[$ {\bf 'r'} $]$ $[$ {\bf 'e'} $]$ $[$ {\bf 'w' } $]$
\end{center}

This line configures the type of debugging information the library
will display. The {\bf 'e'} option specifies that error messages
are to be printed. The {\bf 'w'} option specifies that warning
messages are to be printed. The {\bf 'r'} specifies regular 
tracing information to be printed, and {\bf 'f'} specifies fine
tracing information to be output. Any combination of these options is
allowed. The default setting is to print warning error messages only.


\subsubsection{Domain Specification}

The separation of the configuration file in domain sections is achieved
through the keyword {\bf domain}. If only a local domain exist, this
keyword may be obviated. The format of a domain specification line is
as follows:

\begin{center}
{\bf 'domain'} $[$ {\em domain-name} $]$
\end{center}

The {\em domain-name} must appear only when the user is specifying a
domain which is not the local domain, that is, an empty {\em
domain-name} specifies the local domain. This configuration line may
also be obviated for the local domain if the local domain is specified
first in the configuration file. If several domains are being
specified and the local domain is not the first domain specified in
the configuration file, then this configuration line must appear
explicitly, just before the specification of the local domain.

\subsubsection{Servers Specification}

This section must appear at the beginning of each domain section in
the configuration file, and it is composed of lines starting with the 
{\bf global} or {\bf ldapserver} keywords.

A line that begins with the keyword {\bf global} specifies search options
that will be assumed by all the LDAP servers specified by the {\bf
ldapserver} lines that follow it, and that do not have specific
settings for the options referred in the {\bf global} line.
The format of a global line is as follows:
	
\begin{center}
{\bf 'global'} $[${\em option-name}{\bf $=$}{\em option-value}$]+$
\end{center}		

Here {\em option-name} can be one of the following server options:
{\bf base}, {\bf scope}, {\bf deref}, {\bf binddn}, {\bf bindpwd},
{\bf timelimit}, {\bf sizelimit}, {\bf attonly}, or {\bf timeout}. 
The {\em option-value} string stands for any allowed value of the
option specified by {\em option-name}.

The {\bf base} option specifies the LDAP distinguished name (DN) to be
used as a base object entry relative to which the LDAP search is to be
performed. This is specially useful when an LDAP server
keeps more than one directory database and the user may want to select
a non-default database to serve its name service requests.
The default value for this option is null and it results in searching
the default database in the specified servers. The base string must be
enclosed in double quotes so that the parser can distinguish the
specification of multiple options from distinguished names including
spaces.

The {\bf scope} option specifies the scope of the search, and it may
assume the following values: {\bf sbase}, {\bf onelevel}, or 
{\bf subtree}. The {\bf sbase} value means that only the base object
of the specified database must be considered for the related search. 
The {\bf onelevel} and {\bf subtree} values specify that only the
children of the base object or the whole subtree under the base object
must be considered for the search respectively. The default
value for this option is {\bf subtree}.

The {\bf deref} option defines how alias objects are to be handled in
searching. This option may assume the following values: {\bf never},
{\bf searching}, {\bf finding}, or {\bf always}. The {\bf never} value
specifies that aliases are not to be dereferenced neither in searching
nor in locating the base object. The {\bf searching} value indicates
that aliases are to be dereferenced while searching in subordinates of
the base object, but not in locating the base object of the search
itself. The  {\bf finding} value indicates to the server that it must
dereference aliases in locating the base object for the search, but not
when searching the subordinates of the base object. The {\bf always} 
value indicates that aliases are to be dereferenced in both, while 
searching and while locating the base object for the search. The
default value for this option is {\bf always}.

The {\bf binddn} and {\bf bindpwd} specify the DN and password to be
used when the client library binds to a server using simple
authentication. The default value for this options is null and it
results in the client library binding as the anonymous user to the
specified servers.

The {\bf timelimit} option specifies the maximum time, in seconds,
allowed by a client for its search requests to be answered.
A value of zero (0) is the default for this option, and it tells the
server that the client has no time-limit restrictions on it search
operations.  The {\bf sizelimit} specifies the maximum number of
entries to be returned as a result of a search. A value of zero (0)
for this option is the default, and it specifies no limit for the size of
server's replies to this client.

The {\bf attonly} is a integer-valued option that specifies if only
the type-names of the attributes must be returned as a result of
search operation (non-zero value) or if attributes' type names and
values are to be returned (zero value). Zero is the default
value for this option\footnote{Currently it does not make sense to set
this option to a non-zero value, but it has been included to support 
future extensions to the LDAP client library.}.

The {\bf timeout} option is an integer-valued option that specifies 
the time interval to be elapsed before deallocating idle sessions. The
value represents the number of 30 seconds cycles that a given session
needs to be idle in order to be closed. The current default value for
this option is two (2) resulting in one minute idle timeouts for
inactive LDAP sessions.

Note that there may not be spaces within an option specification
except for those options that have a distinguished name (DN) as a value, but
different options specifications must be separated by white spaces or 
tabs. Spaces within DN are identified by the parser because DN must be
enclosed in double quotes.

A line in the configuration file that begins with the {\bf ldapserver}
keyword specifies a server to be used by the client library when it
needs to resolve a name service requests. Several server specification
lines may appear in the configuration file. The servers specified in the
configuration file will be tried in the order in which they appear in
the configuration file when the client library fails to connect to
servers specified previously.

The format of a server configuration line is as follows:

\begin{center}
{\bf 'ldapserver'} {\em ( host-name $|$ ip-address )}$[$ {\bf ':'} {\em
port-number} $]$ $[$ {\em option-name} {\bf '='} {\em option-value} $]$*
\end{center}		

The {\em host-name} or {\em ip-address} specify a valid DNS name or a
valid IP address respectively for an LDAP server. If a {\em host-name} is
specified, the name must be locally resolvable, otherwise the specific
server configuration line is discarded.
If no server is specified in a configuration file, the local host will
be assumed as the default server. The {\em option-name} trailing part
of this line is a list of server options separated by white spaces as
specified above for the {\bf global} directive. Values specified here
will override any global option values that may have been specified
previously in the configuration file. 
The {\em port-number} is optional and it specifies the port to be used
by the LDAP client library when connecting to the specific server.

\subsubsection{Schema Specification}

This part of the configuration file contains the specification of the 
schema supported by the specific databases holding information to be
used by the name service daemon. This part of the configuration file
which may be present within each domain specification section, and it
is headed by a {\bf schema} directive. This directive must be placed 
right after the server specification for each domain. The format of
this line is as follows:

\begin{center}
{\bf 'schema'} $[$ {\em schema-name} $]$
\end{center}

The {\em schema-name} is optional and if it does not appear the
schema specification to follow will be named as the current domain.
If within a domain specification this keyword is not used, all 
the mapping rules specified will be grouped under a schema named as
the current domain. All the schema rules contained in {\bf tablemap}
and {\bf format} directive lines found after a {\bf schema} line will be
grouped under a schema name. The first schema specified within a
domain is the default schema for all the servers in that domain.
The user, however, may specify explicitly which schema is to be used
by each server using the {\bf setschema} line. In addition, the
default schema for a given domain may be changed by using the 
{\bf setdefschema} keyword. Schema names have global scope and they
can be reused in domains different from those in which they are
defined.

The {\bf setschema} line has the following format:

\begin{center}
{\bf 'setschema'} {\em schema-name} $($ {\em server-name } $)+$
\end{center}

The schema referred by {\em schema-name} and the servers referred by 
{\em server-name } must have been previously declared in the
configuration file. The list of server names may contain DNS names or
IP addresses that must be identical to those used in the server
declaration. Items in the list of servers must be separated by white
spaces or tabs.

The {\bf setdefschema} line has the following format:

\begin{center}
{\bf 'setdefsetschema'} {\em schema-name}
\end{center}

It may appear only after the definition of the schema named
{\em schema-name} and as stated previously sets the default schema for
all the servers without explicit schema specification.

The schema definition section must follow the schema declaration
section. This part of the configuration file specifies how the
different name service requests generated by the operating system will
be mapped into LDAP protocol queries and how the responses to these
queries are mapped back into the related file formats.

The {\bf tablemap} directive line has the following format and it
specifies how name service requests are to be mapped into an LDAP
search filters for the schema being defined:

\begin{center}
{\bf 'tablemap'} {\em name-of-table} {\em LDAP-filter}
\end{center}

The {\em name-of-table} specifies the NIS map being mapped to the LDAP
search query specified by {\em LDAP-filter}. The {\em LDAP-filter} is
an ASCII string as specified in RFC-1960, but it includes {\em \%s}
signs in places where the keys for the search are to be substituted. 

A format line describing how the information will be presented to the
user for the specific Unix configuration file (i.e. protocols) must precede
the {\bf tablemap} lines for the NIS maps  (i.e. protocols, protocols.byname,
protocols.bynumber ) related to the specific configuration file.
If a {\bf tablemap} line is repeated for a given table, the second line
specified for that table is ignored. Mapping lines that appear before
their related format specification in the configuration file are ignored.
Format and {\bf tablemap} lines only need to be specified for those
configuration files for which the LDAP client library will be used, as
specified in the nsswitch.conf file. For example, if the ldap protocol
is only specified for the {\em host} configuration file in nsswitch.conf, 

\begin{center}
{\bf hosts:                  nis ldap dns files}
\end{center}


Then, only the format for hosts and the {\bf tablemap} lines for the hosts,
hosts.byaddr, and hosts.byname tables  need to be included in the
ldap-ns.conf file.


The {\bf format} line has the following format:

%FIX THIS MESS

\begin{center}
{\bf 'format'} {\em table-name}\\
{\bf '('} {\em sep-char} {\bf ')'}
{\bf '$<$'} ( ( {\em attr-name} ( $[$ {\bf '$[+]$'} $|$ {\bf '$[*]$'} $]$ )
$[$ {\bf '$\{$'} {\em sep-char} {\bf '$\}$'} ) $]$
$[$ {\bf '('} {\em sep-char} {\bf ')'} $]$\\
( {\bf ','} $|$ {\bf '$|$'} ) )*\\
( {\em attr-name} ( $[$ {\bf '$[+]$'} $|$ {\bf '$[*]$'} $]$ )
$[$ {\bf '$\{$'} {\em sep-char} {\bf '$\}$'} ) $]$
$[$ {\bf '('} {\em sep-char} {\bf ')'} $]$
{\bf '$>$'}
\end{center}

This line specifies how the attributes returned by an LDAP search must
be formatted to be passed to the user. The sep-char denotes the
character to be used in separating the values of different attributes
or multiple values of a multivalued attribute. For example, if we are
specifying the format for the passwd file, the global separator, which
is specified by the first sep-char that appears between parenthesis in
a format line, must be colon (:). The rest of the text in a format
line specifies the order in and from in which attributes are to be
placed when passed to the user. The + sign specifies that if the
attribute is multi-valued all the values retrieved must be printed in
that position. Multiple values of an attribute for which the + sign
was specified will be separated with the character specified for that
attribute, that is, the optional char value between braces immediately
following the value of the attribute or the global separator in case
the specific separator is not specified. For example, when listing the
hosts file, the associate domain name may be an attribute holding all
the possible alias for a host and all of them will have to be printed
just after the related IP address when passed to the user.

The * sign must appear after those multi-valued attributes which need
to be printed in different lines. There can only be one such attribute
in a given format line. As an example, suppose that services are stored
in entries in which the protocol attribute may have multiple values,
that is, as many as protocols are used to provide the same
service. When presented to the user, this multiple values of protocols
have to be presented in different lines which have the same values for
all the other attributes. For example loc-srv 135/tcp, loc-srv 135/udp
instead of loc-srv 135/tcp,udp.  If an attribute name, which can be
multi-valued is specified without the + or * switches only one value
will be printed.

If the name of an attribute in the format line is followed by a
separator character between parenthesis, then that character is to be
used in separating the value of that attribute from the value of the
next attribute in the final result.

Different attribute names in the format line are separated by commas
or an 'or' ($|$) sign. If two attributes are separated by an 'or'
sign, the value of the second attribute is optional and will only be
printed if the previous attribute has a null value. This
feature is used to format the output of the GECOS attribute in the
password field. The attribute holding the GECOS entry will be
printed if it has a non-null value. If the GECOS attribute has a null
value the Name attribute will be printed\footnote{ We assume the
schema used stores the information in two different attributes. }.

\subsubsection{Example: A Protocol Database}

Assume we want to use the LDAP client library to resolve name service 
requests related to the {\em protocols} configuration file in the local
domain. Additionally, assume two slightly different schemas are used,
each one with entries composed of two attributes. The name of the attributes
in the first schema are: common-name (cn) and ipProtocolNumber. The
attribute naming for the second schema is: common-name (cn) and
ProtoNumb. 

The first step to take to configure this functionality is to tell the 
name-service daemon to use the ldap library for this type of
requests. In order to do this, the file nsswitch.conf must be edited
and the following line  must be added:

\begin{center}
{\bf protocols:             ldap files}
\end{center}

Note that this line must replace any previous configuration line
within this file used for protocols.

After this, we must create a configuration file for the LDAP library in
/etc/ldap-ns.conf. The file should look as follows:

\begin{verbatim}
01:# Local domain declaration:
02:
03:# We obviate the 'domain' keyword because this is the local domain
04:
05:# Global settings valid for all servers
06:
07:global timelimit=60
08:
09:# Alternative servers to contact
10:
11:ldapserver tiger.engr.sgi.com:777 sizelimit=128
12:ldapserver lion.corp.sgi.com base="organizationalUnitName=Engineering,
  o=Silicon Graphics Inc. , c=US"
13:ldapserver eagle.cray.com  base="o=Silicon Graphics Inc. , c=US"
14:
15:# We obviate the schema declaration because this is the default 
16:# schema for the local domain.
17:
18:# Mapping and formatting rules
19:
20:format ( )<cn,ipProtocolNumber,cn[+]>
21:
22:tablemap protocol.byname (&(objectClass=ipProtocol)(cn=%s))
23:tablemap protocol.bynumber (&(objectClass=ipProtocol)(ipProtocolNumber=%s))
23:tablemap protocol (objectClass=ipProtocol)
25:
26:# Declarations for an additional domain:
27:
28:domain nevada
29:
30:# Global options for this servers are the same as above unless
31:# the global keyword is reused here.
32:
33:# Servers for this domain:
34:
35:ldapserver tahoe.corp.sgi.com
36:ldapserver vegas.corp.sgi.com
37:ldapserver reno.corp.sgi.com
38:
39:# Schema specification (schema-name=nevada by default):
40:
41:schema
42: 
43:# Mapping and formatting rules
44:
45:format ( )<cn,ProtoNumb,cn[+]>
46:
47:tablemap protocol.byname (&(objectClass=Proto)(cn=%s))
48:tablemap protocol.bynumber (&(objectClass=Proto)(ProtoNumb=%s))
49:tablemap protocol (objectClass=Proto)
50:
51:# set a different schema for the vegas server in this domain
52:
53:setschema ``'' vegas.corp.sgi.com
54:
\end{verbatim}

The line numbers followed by the colons are only included for didactic
purposes and they should not appear in the real configuration file. 
Line 7 specifies that the default timeout for the search operation
must be set to 60 seconds, for all the servers that do not have a
specific timelimit in their configuration lines. In this case all the
servers will assume this value because no server has an specific
timelimit set. The first server specification line is contained in
line 11. This line states that there is an LDAP server in host
tiger.engr.sgi.com, listening at the non-default port number 777. The
sizelimit option in this line specifies that server in host
tiger.engr.sgi.com must restrict its response size to a maximum of
128 entries when queried by this client. Lines 12 and 13 specify two
alternative LDAP servers, both of these with a specific search base
defined.

Line 20 specifies how the information must be presented to the user,
that is the format in which it would appear in a regular Unix file,
/etc/protocols. Elements must be separated by spaces as specified by
the initial white space enclosed in parenthesis. The first attribute in the
format line specifies that the main value of the common name for the
protocol must be written first, followed by the protocol number
attribute. If the protocol has more than one name (i.e.
the cn attribute is multi-valued) all the extra values or aliases for the
protocol must be printed at the end of the line separated by spaces
since no specific separator is specified for values of the attribute.

Lines 22 through 23 specify the LDAP query corresponding to each
NIS map. For example, to get a protocol entry by name, a filter that
matches the protocol name given with the common name attribute of our
database schema must be used. The filter specified for the protocols 
map matches all the entries used to store protocol information in our
database.

Line 28 defines a new domain with three servers defined in lines 35
through 37. A default schema for the {\em nevada} domain is declared
in line 41, note that since the name parameter for this schema is
missing it assumes the name of the domain, that is, {\em nevada}.
The schema is defined in lines 45 and 47 through 49. Finally, the
schema used in the local domain is used for the server with name
vegas. Note that the basic difference between the two schemas is the
name of the attribute used to store the protocol number.
















